\chapter{Struktur}
  \section{Spuren}

      Ein Projekt besteht aus Spuren analog einer analogen Bandmaschine.
      Die Spuren enthalten Midi- Audio- oder Controllerdaten.
      Spuren werden in \M\ in zwei Formen dargestellt:

      \blank[big]
      \Input{Arranger:} im Arranger werden alle Spuren vertikal in einer
            Liste dargestellt

      \Input{Mixer:} der Mixer zeigt alle Spuren in einer horizontalen
            Darstellung\par

      \blank[big]

      Tempo und Taktart sind interne Spuren, die nicht im Arranger oder
      Mixer gezeigt werden.

  \section{Parts}

      Midi- und Audiospuren können auf der Zeitachse in Parts unterteilt
      werden. Parts enthalten Midi- Audio- oder Conrollerdaten.

      Folgenden Operationen sind mit Parts möglich:

      \blank[big]
      \Input{verschieben} Parts können auf der Zeitachse und in andere
            Spuren des gleichen Typs verschoben werden.
      \Input{kopieren} klar!
      \Input{klonen} erzeugt einen Part, der sich die Midi- Audio- oder
            Controllerdaten mit dem Quellpart teilt. Wird ein Event in
            einem Clone-Parts verändert, so verändern sich auch alle
            Clones.

      \Input{Stumm} Parts können stumm geschaltet werden. Stumme Parts
            werden im Arranger grau dargestellt.

      \par
      \blank[big]



