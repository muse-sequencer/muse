\section{Midi Recording}
      Auf der ersten Tour werden wir mit \M\ eine Midi Spur
      aufzeichnen.

      Vorbereitungen: JACK starten, Keyboard einschalten. 
      Ist das Keyboard per Midi richtig angeschlossen? 
      Dann kanns losgehen. Wir starten \M\ und es erscheint das
      zuletzt bearbeitete Projekt so wie wir es verlassen haben.

      Um ein neues Projekt anzulegen clicken wir auf das Projekt
      Icon und der Projektdialog erscheint:

      \Fig{select_project}

      Wenn \M\ zum allererstenmal gestartet wird, erscheint der
      Projektdialog natürlich sofort, da es ja kein letztes Projekt
      gibt. Zum Projektdialog gelangen wir auch über den Projekmenü
      Eintrag ''Öffnen'' oder wer es ganz eilig hat tippt einfach
      {\tt Ctrl+O}.

      In der ''Projekt'' Eingabezeile geben wir nun einen Projektnamen
      für unser erstes Projekt ein und bestätigen dann durch clicken
      des ''Ok'' Buttons.

      \M\ fragt nun nach einem Template, mit dem das Projekt
      initialisiert werden soll:

      \Fig{select_template}

      Wir selektieren kein Template und clicken einfach ''Ok''.
      \M\ zeigt dann ein leeres Projekt:

      \Fig{main0}

      Zunächst erzeugen wir einen ''MidiInput'' Track und checken,
      ob das Keyboard richtig angeschlossen ist:

      
